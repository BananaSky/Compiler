\documentclass{article}

\usepackage{indentfirst}
\usepackage{setspace}
\doublespacing

\title{Fully Generalized Compilation}
\date{01-26-2017}
\author{Lucas Saldyt}

\begin{document}

\maketitle
\pagenumbering{gobble}
\newpage
\pagenumbering{arabic}

\section{Abstract}

Assuming that two languages are \textit{isomorphic}, one syntax can be converted to another if given an adequate description of each syntax's grammar. At an abstract level, this conversion is entirely practical. Progtran is an investigation into the feasibility of a \textbf{Fully Generalized Compiler}, a multi-language source-to-source compiler. Given source code in one language, a description of its grammar \textbf{A} , and a description of an output language's grammar \textbf{B}, can \textbf{A} be converted to \textbf{B} such that the output in \textbf{B} is indistinguishable from a program originally written in \textbf{B}?

\section{Introduction}

A \textbf{Full Generalized Compiler} is a tool for programming language researchers, providing a \textit{dynamic framework} for easily modifying syntax of input and output languages. A new programming language can even be fully implemented without recompiling the software. However, Progtran also uses this concept (user defined dynamic grammars) to re-implement existing source-to-source compilation tools, such as \textbf{Cython} or \textbf{2to3}.

\section{Methods}
\subsection{Grammars}
\subsection{Constructors}
\subsection{AST Transformations}


\section{Results}

\end{document}
